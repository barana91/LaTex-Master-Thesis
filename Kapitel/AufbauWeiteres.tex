
\thispagestyle{fancy}

\section{Bestimmung der Degradation des UV Quarzglases}
%
\begin{figure}[htb]
  \centering
  \begin{minipage}[t]{0.49\linewidth}
      \centering
      \includegraphics[width=\linewidth]{Bilder/uvsilicaDegradation.png}
      \caption{Vom Hersteller angegebene wellenlängenabhängige Transmission vor und nach Degradation durch Bestrahlung.}
      \label{fig:degra}
  \end{minipage}
\end{figure}
\noindent
Da die Messung der Anregungsleistungdichte erfolgt, bevor das Laserlicht die Probe durch das UV-Quarzglas im Kryostaten trifft, ist es wichtig den Transmissionsverlust zu bestimmen, um die realen Werte für die Anregungsleistungsdichte zu kennen (Die Anregungsleistungsdichte, die bei der Probe ankommt). Der Kryostat besitzt vier Fenster, bestehend aus UV-Quarzglas, das besonders transparent im UV-Wellenlängenbereich ist. Durch diese Fenster dringt das Laserlicht in den Probenhalter ein. Von diesen Fenstern war und ist eines in dauerhaftem Gebrauch. Wie in Abbildung \ref{fig:degra} zu sehen ist, weisen die Fenster aber mit der fortlaufender Bestrahlung Degradation auf. So nimmt laut Hersteller durch Degradation die Transmission von 90 Prozent bis auf ca. 75 Prozent für eine Wellenlänge von 193 nm ab.
\newline
Um nun den Transmissongrad zu bestimmen, wurde der Probenhalter aus dem Kryostat entfernt, damit das Laserlicht ungehindert durch zwei parallel liegende Fenster durchdringen kann, um so auch die Anregungsleistungsdichte des Laserlichtes nach durchdringen des letzten Fenster messen zu können. So konnte die Anregungsleistungsdichte vor dem Eintreten und nach dem Austreten in den Kryostaten bestimmt werden. 
\newline
Dies wurde einmal bei den parallel liegenden unbenutzten Fenstern durchgeführt. Davon ausgehend, dass beide Fenster, da unbenutzt, den gleichen Transmissionsgrad haben, kann darauf zurückgeschlossen werden, dass ein unbenutztes Fenster einen Transmissionsgrad von 59 Prozent aufweist. Dies weicht um ca. 10 Prozent von der Herstellerangabe ab, die aber nicht die Reflektion im Kryostaten miteinbezieht. 
%
\begin{figure}[htb]
  \centering
  \begin{subfigure}{0.40\textwidth}
    \centering
    \includegraphics[width=0.9\linewidth]{Bilder/uvsilicavergleich.pdf}
    \caption{PL Intensität in Abhängigkeit der Anregungsleistungsdichte mit den benutzten und unbenutzten Fenstern}
    \label{fig:sub1}
  \end{subfigure}%
  {\LARGE$\xrightarrow{\cdot 0,44}$}
  \begin{subfigure}{0.40\textwidth}
    \centering
    \includegraphics[width=0.9\linewidth]{Bilder/uvsilicaVergleichSkaliert.pdf}
    \caption{Die Anregungsleistungsdichte des unbenutzten Fensters mit den rechnerisch bestimmten 0,44 skaliert}
    \label{fig:sub2}
  \end{subfigure}
  \caption{}
  \label{fig:vergleichSkaliert}
\end{figure}
\newline
Um den Transmissionsgrad des benutzten Fensters zu bestimmen, wurde die Transmission durch das benutzte und dem parallel liegende unbenutzte Fenster gemessen. Mit dem Wissen, dass der Transmissionsgrad durch das unbenutzte Fenster bei 59 Prozent ($T_{unb} = 0,59$) liegt, kann die Transmission durch das benutzte Fenster auf 26 Prozent ($T_{ben} = 0,26$) berechnet werden. 
Davon ausgehend, dass die Transmission bei höheren Wellenlängen (Emission) gleich und bei beiden Fenstern ähnlich ist,
%
\begin{equation}
  F_{skal} = \frac{ T_{ben} }{ T_{unb} } = 0,44\label{eq11}
\end{equation}
%
kann der Skalierungsfaktor mit für die Anregungsleistungsdichte auf 0.44 bestimmt werden. Dies bedeutet, dass durch Degradation die Transmission auf 44 Prozent der ursprünglichen Transmission gesunken ist.
\newline
Dies bestätigt sich auch durch die Gegenüberstellung in  Abbildung \ref{fig:vergleichSkaliert}. Somit ist durch die zeitliche Degradation die Transmission auf 44 Prozent der ursprünglichen gesunken. Dies bestätigt im Umkehrschluss, dass von der ausgehenden Anregungsleistungsdichte, die vor dem Kryostaten gemessen wird, durch die geringe Transmission der Fenster und Reflektion im und außerhalb des Kryostaten nur ca. 26 Prozent des vom Laser emittierten Lichtes bei der Probe ankommen. 