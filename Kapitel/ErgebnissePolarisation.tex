\thispagestyle{fancy}

\chapter{Untersuchung der optischen Polarisation an AlGaN MQWs mit Photolumineszenzspektroskopie}
\label{chap:pol}
\section{Einleitung}
Um die Polarisation und den Kreuzungspunkt der Simulationen von Christoph Reich experimentell zu \"uberpr\"ufen, wurden die Polarisation von zwei Probenserien mit Hilfe von Photolumineszenz-Spektroskopie untersucht. Die untersuchten Probenserien unterteilen sich in eine Serie mit Variation des Al-Gehalts (Serie A) in den QWs und eine QW-Dicken-Variation (Serie B).
Alle Proben wurden bei Raumtemperatur (300K) untersucht und die Emission wurde aus der Kante der Probe gemessen, um das Verhältnis zwischen TE- und TM-polarisertem Licht zu bestimmen. Weil es m\"oglicherweise Auswirkungen des ELO auf die Polarisation gibt, wurde der Einfluss des ELO mit untersucht. 

\section{Variation des Al-Gehalts in den QWs}

Die Untersuchung der Al-Variations Serie, dient dem Zweck, anhand eines variierenden Al-Gehalts in den QWs und einem festen Al-Gehalt in der Barriere den \"Ubergang von TE zu TM (siehe Abb. \ref{fig:simuchr} in Kapitel \ref{chap:polgrund}) bei fester QW-Dicke experimentell zu \"uberpr\"ufen. Dazu wurden vier Proben auf ELO-AlN gewachsen, mit einer darauf folgenden AlN(100\%) Buffer-Schicht. Auf die Buffer-Schicht folgt zuletzt die aktiven Zone, mit einem zwischen der ersten und letzten undotierten AlN-Barriere eingebetteten dreifach $Al_{x}Ga_{1-x}N$ QWs mit einer Dicke von $1.5nm$ und dazwischen AlN-Barrieren einer Dicke von $5 \thinspace nm$. Der Aluminium-Gehalt der QWs wurde variiert mit x= 60 \%, 68 \%, 73 \%, 81 \%. 
Der Einfluss des unterschiedlichen Al-Gehalts auf die Emissionsenergie ist in Abb. \ref{fig:alvariationSpektrum} zu erkennen. Die kleinste Wellenl\"ange hat Probe A:4 mit einem Al-Gehalt von $81\%$, die theoretisch ausreicht, um im Vergleich mit den anderen Proben zumindest einen Abfall des Polarisationgrades der TE-Polarisation zu erkennen.
%
\begin{figure}[htb]
  \centering
  \begin{minipage}[t]{0.49\textwidth}
    \centering
    \includegraphics[width=\textwidth]{Bilder/spektrenAlvariation.pdf}
    \caption{PL-Spektren der Serie A. Die Emission verschiebt sich mit steigendem Al-Gehalt in den QWs hin zu kleineren Wellenl\"angen durch die steigende Bandl\"uckenenergie. }
    \label{fig:alvariationSpektrum}
  \end{minipage}
	\hfill
  \begin{minipage}[t]{0.49\textwidth}
    \centering
    \includegraphics[width=\linewidth]{Bilder/polarisationAlvariation.pdf}
    \caption{Ergebnisse der Polarisationsmessungen an Serie A mit den Polarisationsgraden abh\"angig vom Al-Gehalt des QW und zus\"atzlich mit den Messwerten in Abh\"angigkeit der ELO-Richtung. }
    \label{fig:alvariationPolarisation}
  \end{minipage}
\end{figure}
%
Dazu wurden Proben vertikal und horizontal zur ELO-Richtung positioniert und gemessen. Abbildung \ref{fig:alvariationPolarisation} zeigt die Ergebnisse der Polarisationsmessungen. So zeigen die Proben A:1 und A:2 mit einem Al-Gehalt von $60\%$ und $68\%$ TE-Polarisation. Der Grad der Polarisation ist zus\"atzlich noch abh\"angig von der Positionierung der ELO-Streifen. So haben die Proben A:1 und A:2 vertikal zur ELO-Richtung Polarisationsgrade von $\rho = 0,33$ und $\rho = 0,30$ und parallel zur ELO-Richtung deutlich geringe Polarisationsgrade mit $\rho = 0,22$ und $\rho = 0,25$. Die Proben A:3 und A:4 mit $73\%$ und $81\%$ Al-Gehalt weisen TM-Polarisation auf mit Polarisationsgraden von $\rho = -0,09$ und $\rho = -0,05$ vertikal zur ELO-Richtung und $\rho = -0,11$ und $\rho = -0,14$ parallel zur ELO-Richtung. Es zeigt sich demzufolge, dass sich die Polarisation mit steigendem Al-Gehalt von TE- hin zu TM-Polarisation durch die Neuordnung der Valenzb\"ander \"andert. Der Wechsel findet bei einer Wellenl\"ange von ca. $240nm$ statt und ist in guter \"Ubereinstimmung mit den Simulation (siehe Abb. \ref{fig:simuchr}). \"Uberdies ist eine Abh\"angigkeit der Ausrichtung der ELO-Streifen zu beobachten. M\"ogliche Erkl\"arungen w\"aren, dass es durch Brechungsindexwechsel vom Freiraum des ELO zum Kristall, zu Reflektion des emittierten Lichtes und damit zu Interferenzerscheinungen kommt oder das ELO die Verzerrung im Kristall so beeinflusst, dass die f\"ur die Simulation angenommene biaxiale Verzerrung nicht mehr zutrifft. 

\section{Variation der QW-Dicke}
\begin{figure}[htb]
  \centering
  \begin{minipage}[t]{0.49\textwidth}
    \centering
    \includegraphics[width=\textwidth]{Bilder/spektrenQWvariation.pdf}
    \caption{PL-Spektren der Serie B. Die Emission verschiebt sich mit steigender QW-Dicke hin zu gr\"oßeren Wellenl\"angen durch den QCSE und Confinement.  }
    \label{fig:qwvariationSpektrum}
  \end{minipage}
	\hfill
  \begin{minipage}[t]{0.49\textwidth}
    \centering
    \includegraphics[width=\linewidth]{Bilder/polarisationDickenvariation.pdf}
    \caption{PL-Spektren der Serie B. Die Emission verschiebt sich mit steigendem Al-Gehalt in den QWs hin zu kleineren Wellenl\"angen durch die steigende Bandl\"uckenenergie. }
    \label{fig:qwvariationPolarisation}
  \end{minipage}
\end{figure}
\noindent
Die Untersuchung der QW-Dicken-Variations-Serie, dient dem Zweck, anhand der variierenden QW-Dicke bei einem festen Al-Gehalt in QW und Barriere die Änderung in der Polarisation(siehe Abb. \ref{fig:simu1chr} in Kapitel \ref{chap:polgrund}) experimentell zu \"uberpr\"ufen. Dazu wurden vier Proben auf ELO-AlN gewachsen, mit einer darauf folgenden AlN(100\%) Buffer-Schicht. Auf die Buffer-Schicht folgt zuletzt die aktive Zone, mit einem zwischen der ersten und letzten undotierten AlN-Barriere eingebetteten dreifach $Al_{0.6}Ga_{0.4}N$ QWs und dazwischen $Al_{0.81}Ga_{0.19}N$-Barrieren mit einer Dicke von $5nm$. 
Der Einfluss der unterschiedlichen QW-Dicke auf die Emissionsenergie ist in Abb. \ref{fig:qwvariationSpektrum} zu erkennen.
\newline
Die Ergebnisse der Polarisationsmessung zeigen eine eindeutig dominante TE-Polarisation, die auch abh\"angig von der ELO-Richtung ist. Die Probe B:2 zeigt die h\"ochste TM-Polarisation mit einem Polarisationsgrad von $\rho=+0,72$ vertikal zur ELO-Richtung. 
\newline Es zeigt sich, wie in der Simulation
in Abb. \ref{fig:simu1chr} zu sehen, dass mit einem Al-Gehalt von $60$\% in den QWs, der Polarisationsgrad bei den Proben mit $1$ und $4 \thinspace nm$ QW-Dicke am niedrigsten und bei $2 \thinspace nm$ am h\"ochsten ausf\"allt . Somit best\"atigt der Verlauf die Simulationen in Bezug auf den zu erwartenden Trend, wo am Rand zum Wechsel (bei $1$ und $4 \thinspace nm$) von TE zu TM der geringste Polarisationsgrad zu erwarten ist.
