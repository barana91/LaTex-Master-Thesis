
\thispagestyle{fancy}


\section{Rekombinationsmechanismen}

In der Photolumineszenzspektroskopie wird Licht als Anregungsquelle für die Anregung von Halbleitermaterialien für die Erzeugung eines Elektron-Loch-Paars. Dazu wird ein Elektron aus dem Valenzband in das Leitungsband angehoben und dabei ein Loch zurückgelassen. Die Elektronen relaxieren anschliessend sehr schnell in das Minimum des Leitungsbandes. Dieser befindet sich im Fall von AlGaN im reziproken Raum am gleichen $\vec{k}$ Vektor, dem $\tau$ -Punkt. Das macht das Materialsystem AlGaN zu einem direkten Halbleiter, was von besonderem Vorteil ist. Denn ein direkter Bandübergang, ist Grundlage für eine Effizienz 