
\section{Bestimmung der internen Quanteneffizienz}


Die aktive Region einer idealen LED würde für jedes injizierte Elektron jeweils ein Photon aussenden. 
Das bedeutet, die IQE die nach \cite{schub} wie folgt definiert ist
\begin{equation}
    IQE = \frac{ \footnotesize \text{Anzahl der Photonen die von der aktiven Zone emittiert werden pro Sekunde}}{ \footnotesize \text{Anzahl der Elektronen die in die LED injiziert werden pro Sekunde}}
\end{equation}
müsste den Wert $1$ annehmen. Die IQE kann somit analog beschrieben werden als Verhältnis von radiativer Rekombination und der effektiven Rekombination. Beschrieben mit Ratengleichungen und mit \ref{eq:iqe1} ist die IQE in ihrer einfachsten Form somit
\begin{equation}
    IQE = \frac{B \cdot n^2}{A \cdot n + B \cdot n^2 + C \cdot n^3} = \frac{R_{rad}}{R_{eff}}
\end{equation}
Die IQE kann mit Hilfe der Photolumineszenzspektroskopie bestimmt werden, in dem angenommen wird, dass keine thermisch aktivierten Defekte bei Raumtemperatur vorhanden sind
\begin{equation}
    A \propto e^{\frac{-E_{activation}}{kT}}
\end{equation}
Mit dieser und der Annahme das keine Auger Rekombination ($ C \cdot n^3 $) auftritt, ist die IQE bei Tieftemperatur ($ \propto 5K$) gleich 1. Somit kann die IQE beschrieben werden
\begin{equation}
    IQE = \frac{\text{Integrierte PL Intensität (T)}}{ \text{Integrierte PL Intensität } (T \rightarrow 0 K) }
\end{equation}
Als Quotient der integrierten PL Intensität bei Temperatur T und integrierter PL Intensität bei Tieftemperatur ($5K$). Die IQE ist folglich abhängig von der Temperatur, da der Paramater A für die SRH-Rekombination temperaturabhängig ist [Abb. \ref{fig:abha}]. 
Um also die IQE bei Raumtemperatur zu bestimmen, wird das Spektrum einer Probe bei 5K und 300K bei ansonsten möglichst gleichen Bedingungen aufgenommen. Die Intensität in Abhängigkeit der Wellenlänge wird interpoliert, dann integriert und dann das Verhältnis berechnet.
%
\begin{figure}[tb]
    \centering
    \begin{minipage}[t]{0.49\linewidth}
        \centering
        \includegraphics[width=\linewidth]{Bilder/IQEohneDotierungVerschAParams.pdf}
        \caption{Die Grafik zeigt die Abhängigkeit der internen Quanteneffizienz von der Ladungsträgerdichte für feste Paramater B und C. Der Paramater wird A wird variiert mit 9 verschiedenen Werten von $0 s^{-1} $ bis $10^9 s^{-1}$ ~\cite{semreich}.}
        \label{fig:abha}
    \end{minipage}% <- sonst wird hier ein Leerzeichen eingefügt
    \hfill
    \begin{minipage}[t]{0.49\linewidth}
        \centering
        \includegraphics[width=\linewidth]{Bilder/IQEmitDotierungVerschAParams.pdf}
        \caption{}
        \label{fig:abha1}
    \end{minipage}
\end{figure}
\vspace{1cm}
\raggedright
\newpage
%
Das ABC-Modell stellt aber nur eine Vereinfachung dar und berücksichtigt nicht alle vorkommenden Effekte wie beispielsweise Lokalisierung, Screening durch Ladungsträger und Dotierung. 
Der Effekt der Dotierung spielt dabei eine besonders wichtige Rolle, da eine Silizumdotierung üblich in UV-Leds ist und einen großen Einfluss hat.
Nach \cite{schub} kann gezeigt werden, dass die Ratengleichungen, mit der Annahme einer Dotierung für die 
die Radiative Rekombination sich ändert und soll nun hergeleitet werden:
\\newline
Jeder dotierte oder undotiere Halbleiter hat zwei Arten von Ladungsträgern, Elektronen und Löcher.
Im Gleichgewicht, bedeutet ohne externe Anregung durch Absorption von Licht oder Injektion von Elektronen, ist das Produkt von Elektronen- und Lochkonzentration eine konstante Größe.
\begin{equation}
    n_0 \cdot p_0 = n_i^2
    \label{eq:constant}
\end{equation}
Hierbei sind $n_0$ und $p_0$ die Elektron- und Lochkonzentration unter Gleichgewichtsbedingung und $n_{i}$ damit die intrinsische Ladungsträgerkonzentration.
Werden zusätzlich die durch Anregung erzeugten Ladungsträger betrachtet, so ist die Gesamtladungsträgerkonzentration gegeben als Summe der Anregungs- und Gleichgewichtsladungsträger. 
\begin{equation}
    n_{ges} = n_0 + n \medspace \text{und} \medspace  p_{ges} = p_0 +  n 
\end{equation}
Hierbei sind $ n$ und $p$ die Anregungsladungsträger. 
Die Anzahl der stattfindenden Rekombination zwischen Elektronen und Löchern sind direkt proportional zur Elektronen-und Ladungsträgerkonzentration, so gilt, $R \propto n \cdot p $. Mit einer Proportionalitätkonstante, wird Rekombinationrate pro Zeit und Volumen definiert als
\begin{equation}
    R = - \frac{dn_{ges}}{dt} = - \frac{dp_{ges}}{dt} = B \cdot n_{ges} \cdot p_{ges}
\end{equation}
Weil Elektronen und Löcher bei Anregung paarweise erzeugt werden und verschwinden (durch Rekombination), gilt
\begin{equation}
    \label{eq:gleich}
    n(t) =  p(t)
\end{equation}
Die radiative Rekombinationrate wird dann mit $p_{0} = 0$ und Gleichung [\ref{eq:gleich}] zu
\begin{align}
\begin{split}
    R_{rad} &= B \cdot (n_0 + n)  \cdot (p_0 + p) ,
    \\
    R_{rad} &= B \cdot (n_0 + n) \cdot (n) ,
    \\
    R_{rad} &= B \cdot n^2 \cdot n \cdot n_0
\end{split}
\end{align}
Dabei beschreibt $n_{0}$ die Ladungsträgerkonzentration durch die Silizumdotierung. 
Somit wird die IQE zu:
\begin{equation}
    IQE = \frac{B \cdot n^2 + B \cdot n \cdot n_{0}}{A \cdot n + B \cdot n^2  + B \cdot n \cdot n_{0}+ C \cdot n^3} 
    \label{eq:dopediqe}
\end{equation}
Und hat einen enormen Einfluss auf die Ordinate, wie in Abb. [\ref{fig:abha1}] zu sehen ist.
%

