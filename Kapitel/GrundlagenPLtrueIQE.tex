
\chapter{Einfluss der Auger-Rekombination auf die IQE}
\label{chap:auger}
\thispagestyle{fancy}
Das bisher benutzte Modell zur Bestimmung der PL-IQE ging davon aus, dass bei Tieftemperatur keine Auger-Rekombination (Gleichung) vorkommt, aber wie in Abb. \ref{fig:auger5k} deutlich zu erkennen, nimmt der Verlauf der PL-Intensität in doppeltlogarithmischer Darstellung gegenüber der Anregungsleistungsdichte (die direkt proportional zur Ladungsträgerdichte ist), im Bereich höherer Anregungsleistungen (entsprechend höheren Ladungsträgerdichten) deutlich ab und weist keinen linearen Verlauf mehr auf, den er aber durch eine allein quadratische Abhängigkeit (in doppeltlogarithmischer Darstellung linear) haben sollte. 
\newline
\begin{figure}[ht]
    \centering
    \begin{minipage}[t]{0.49\linewidth}
        \centering
        \includegraphics[width=\linewidth]{Bilder/AugerBei5K.pdf}
        \caption{Die Grafik zeigt die integrierte Intensität bei Tieftemperatur ($5 K$) in Abhängigkeit der Anregungsleistungdichte. In doppeltlogarithmischer Darstellung müsste die integrierte Intensität wegen $R = B \cdot n^2$ linear steigen (schwarze Linie).}
        \label{fig:auger5k}
    \end{minipage}% <- sonst wird hier ein Leerzeichen eingefügt
    \hfill
    \begin{minipage}[t]{0.49\linewidth}
        \centering
        \includegraphics[width=\linewidth]{Bilder/NormierteKorrgierteIQE5K.pdf}
        \caption{Die Grafik zeigt die normierte korrigierte IQE bei 5K nach Gleichung \ref{eq:iqetrue5k}}
        \label{fig:trueiqe}
    \end{minipage}
\end{figure}
\vspace{0.1cm}
\noindent
\newline
Dies zeigt, dass die IQE nicht, wie bisher angenommen, immer bei 100 Prozent liegt, sondern auch bei Tieftemperatur Anregungsleistungsdichte abhängig ist. Das dieser Verlustmechanismus ähnlich wie bei InGaN/GaN auf Auger-Rekombination basiert wurde von Nippert et al. bestätigt \cite{doi:10.1063/1.4965298}. 
Um dies zu berücksichtigen, wird nun die IQE bei 5K definiert als:
\begin{equation}
    IQE_{corr}(T = 5K, P_{exc}) = \frac{ \frac{I_{pl}(T,P_{exc}) }{P_{exc} } } { I_{norm}}
    \label{eq:iqetrue5k}
\end{equation}
mit dem Maximum der Verhältnisse von integrierter PL-Intensität zu Anregungsleistungdichte als  Normierungsfaktor über alle $n$ Anregungsleistungdichten
\begin{equation}
    I_{norm} = \lvert \lvert \sum_{i=1}^{n} \frac{I_{pl}(T,P_{exc,i})}{P_{exc,i}} \lvert \lvert_{max}
    \label{eq:iplnorm}
\end{equation}
\noindent
\begin{figure}[htb]
\centering
    \begin{minipage}[t]{0.49\linewidth}
        \includegraphics[width=\linewidth]{Bilder/korrigierteIQE300K.pdf}
        \caption{Vergleich von korrigierter und unkorrigierter IQE bei 300K nach Gleichung \ref{eq:iqetrue300k} }
        \label{fig:trueiqe300k}
    \end{minipage}
\end{figure}
\noindent
\newline
Wobei $I_{pl}(P_{exc})$ die von der Anregungsleistungsdichte $P_{exc}$ und Temperatur $T = 5K$ abhängige integrierte PL-Intensität ist. Die integrierte PL-Intensität wird durch die Anregungsleistungsdichte dividiert und auf das Maxmimum normiert, so dass das Maximum der IQE bei $5K$ bei der geringsten Anregungsleistungdichte mit der geringsten Auger-Rekombination liegen sollte. Um nun die IQE bei Raumtemperatur zu bestimmen, wird die nach dem alten Verfahren ermittelte IQE multipliziert mit den neu ermittelten Werten passend zur Anregungsleistungsdichte. 
\begin{equation}
    IQE_{corr}(T, A_{exc}) = \frac{IQE(T,A_{exc})}{IQE(5K,A_{exc})} \cdot IQE_{corr}(5K,A_{exc})
    \label{eq:iqetrue300k}
\end{equation}
Die IQE bei $5K$ dient hierbei also als Skalierungsfaktor, der den Einfluss der Auger-Rekombination bei Raumtemperatur korrigiert. Somit fällt im Vergleich insbesondere auf, dass die Skalierung bei den kleinsten Anregungsleistungsdichten den geringsten Einfluss hat und mit steigender Anregungsleistungsdichte kubisch steigt, so dass die nach dem alten Verfahren ermittelten IQE-Werte bei höheren Anregungsleistungsdichten deutlich nach unten korrigiert werden müssen. \newline
Auch wenn mit diesem Verfahren dem Einfluss der Auger-Rekombination entgegengekommen wird, so gibt es noch weitere Fallstricke bei der IQE-Bestimmung. Ein erhebliches Problem steckt in der nicht-resonanten Anregung mit dem ArF-Excimer Laser, denn das Laserlicht wird wegen der hohen Energie auch von allen Schichten und damit insbesondere von den Barrieren absorbiert. Die angeregten Ladungsträger müssen in den QW diffundieren und dort relaxieren, rekombinieren aber potentiell in den Barrieren, was im Spektrum als Barrieren Peak speziell bei Tieftemperatur zu sehen ist.
\newline
Daraus und aus Absorption resultierend scheitern AlGaN-Heterostrukturen die mit einem ArF Laser angeregt werden, eine Sättigung in der IQE zu erreichen \cite{doi:10.1063/1.4965298}. Ohne diese ist es schwer eine echte IQE anzugeben, da die in QWs gelangende Laserstrahlung, von nicht-resonanter Anregung ausgehend, von Materialparametern wie der Absorption und thermischen Diffusion abhängen und somit nicht klar ist, welche Anregungsleistungsdichte tatsächlich in die QWs gelangt. Weswegen ein Vergleich von Proben die sich in ihren Absorptionseigenschaften stark unterscheiden, schwer zu vergleichen sind, wenn die maximale IQE nicht bestimmt werden kann \cite{doi:10.1063/1.5044383}. 

