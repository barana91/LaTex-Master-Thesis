\thispagestyle{fancy}

\section{Optische Polarisation und Valenzbandstuktur}

Durch die Prozessierung und die Flip-Chip-Montage kann Licht nur durch die untere, unbewachsene Seite des Saphir-Substrates ausgekoppelt werden. Die Art und Weise der Lichtauskopplung hat einen bedeutenden Einfluss auf die Extraktionseffizienz und damit auf die externe Quantenffizienz(EQE). Durch die Geometrie bestimmt, hängt die Extraktionseffizienz maßgeblich vom Emissionsprofil ab, so dass Licht welches senkrecht zur Quantenfilmebene abgestrahlt wird, die höchste Extraktionseffizienz aufweist. 
Die Valenzbandstrukturen von AlN und GaN unterscheiden sich hauptsächlich durch die unterschiedlich starke Kristallfeldaufspaltung ~\cite{doi:10.1063/1.117689}. 
Das Valenzband der Gruppe-III Nitride wird in drei Subbänder aufgespalten. 
Das Schwerlochband (engl. heavy hole, HH), das Leichtlochband (engl. light hole, LH) und das Kristallfeldaufspaltungsband (engl. split off, SO). 
In AlN hat die Kristallfeldenergie einen Wert von $-217meV$, so ist das SO-Band das oberste Valenzband und das HH- und LH-Band befinden sich etwa $220meV$ unterhalb des SO-Bandes ~\cite{doi:10.1063/1.2840176}. Bei Raumtemperatur wird, nach der Fermi-Dirac-Verteilung, das oberste Band, das SO-Band, besetzt. 
Und das SO-Band, aus Zuständen des $p_z$ bestehend, emittiert Licht, das transversal magnetisch (TM) polarisiert ist. Die strahlende Rekombination findet demnach überwiegend mit Elektronen und Löchern aus dem SO-Band statt und entsprechend ist das emittierte Licht TM polarisiert.
In GaN ist das HH-Band das oberste Valenzband am $\gamma$-Punkt. Das SO-Band liegt mit $48meV$ unterhalb der Valenzbandkante und ist das tiefste Valenzband. Damit ist die Besetzungswahrscheinlichkeit des HH- und LH-Bandes bei Raumtemperatur am größten. Das elektrische Feld des Lichtes ist senkrecht zur c-Kristallachse und wird durch Übergänge ins HH-oder LH-Band erzeugt ~\cite{doi:10.1063/1.3574025}, die aus Zuständen des $p_x$-und $p_y$-Orbitals bestehen. Bei strahlender Rekombination von Elektronen mit den Löchern im HH-und LH-Valenzband entsteht also überwiegend TE-polarisiertes Licht ~\cite{doi:10.1063/1.3574025}. 
TM polarisiertes Licht kann nicht ausgekoppelt werden, da es nur in der x-y- Ebene (parallel zur Quantenfilmebene) emittiert. Daraus resultierend sinkt die Extraktionseffizienz und damit die EQE. Bei $Al_{x}Ga_{1-x}N$ kommt es mit steigendem Aluminiumgehalt zu einer kontinuierlichen Verschiebung der Valenzbänder, sodass die Lichtemission von hauptsächlich TE-polarisiertem Licht zu TM-polarisiertem Licht ändert. Der Wechsel findet theoretisch bei ca. $10\%$ statt ~\cite{doi:10.1063/1.3675451}.
 