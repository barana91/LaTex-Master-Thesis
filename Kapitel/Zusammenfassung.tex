
\chapter{Zusammenfassung}

\thispagestyle{fancy}


In dieser Arbeit wurden PL-Untersuchungen an AlGaN UVC-Laser- und LED-Strukturen durchgeführt und deren IQE bestimmt. Diese wurde verglichen und die Ursachen für die Unterschiede mit Hilfe verschiedener Messmethoden wie AFM und CL versucht zu ergründen. Weiter wurde das Modell zur Bestimmung der IQE erweitert (Kapitel \ref{chap:auger}), in dem die Annahme es würde keine nicht-radiative Rekombination bei Tieftemperatur geben, fallen gelassen wurde. So findet Auger-Rekombination auch bei Tieftemperatur statt und hat durch die kubische Abhängigkeit von der Ladungsträgerdichte einen nicht zu vernachlässigenden Einfluss auf die IQE. Dies zeigte sich durch mit der Erweiterung einhergehenden deutliche Korrekturen der IQE die zu merklich geringeren Werten führte. Des Weiteren wurde der Einfluss verschiedener Effekte auf die Bestimmung der IQE besprochen und damit zusammenhängende Problematiken beschrieben. Wie bspw. die Überbandanregung die es nicht erlaubt resonant Anzuregen und zum Effekt der thermischen Diffusion führt. 
\newline
In Kapitel \ref{chap:mqw} wurde das MQW-Design von AlGaN UVC-LED Heterostrukturen untersucht, in dem die IQEs zweier Probenserien mit und ohne
dotierte Barrieren und je vier Proben untersucht und verglichen wurden.  
Die Ergebnisse haben gezeigt, dass die QW-Dicke einen eindeutigen Einfluss auf die Rekombination und damit auf die IQE hat. Weiter zeigte sich der Einfluss dotierter Barrieren, so führen diese zu höheren IQEs im Bereich geringer Anregungsleistungsdichten. Die Messungen ergaben, dass QW-Dicken von $1-2 \thinspace nm$ zu den höchsten IQEs führen. Die experimentell ermittelten Ergebnisse wurden von $k \cdot p$-Simulationen von Christoph Reich bestätigt. 
\newline
Die Untersuchungen in Kapitel \ref{chap:offcut} widmeten sich der Untersuchung zweier Probenreihen von optisch gepumpten Laserstrukturen, die aus Rezepten aus zwei unterschiedlichen Serien stammen. Untersucht wurde der Einfluss des Fehlschnittwinkels des Substrates auf die beiden Serien, die sich beide darin unterscheiden, dass sie einmal mit und ohne Übergitter gewachsen wurden. Ziel der Untersuchungen war, den Einfluss des Fehlschnittwinkels auf die IQE zu untersuchen. Für eine tiefergehende Analyse wurden noch AFM und CL-Messungen betrachtet und analysiert und mit der IQE versucht in Zusammenhang zu bringen. 
\newline
So zeigte es sich, dass die Proben mit Übergitter eine eindeutig glattere Oberfläche aufweisen. Ein positiver Einfluss auf die Versetzungsdichte bei einem Fehlschnittwinkel von $0.2 \degree$ konnte dagegen nicht eindeutig geklärt werden. Dazu wurde nochmal auf die bestehenden Grenzen bei der Bestimmung der IQE eingegangen und erläutert. Das Fazit aller zusammengetragenen Untersuchungen war, dass die Ergebnisse für eine Verwendung von Übergittern mit einem Fehlschnitt von $0.1\degree$ sprechen.
\newline
Kapitel \ref{chap:pol} widmete sich der Untersuchung der Polarisation und der experimentellen Überprüfung des Kreuzungspunktes der Simulationen von Christoph Reich. Dazu wurden die Polarisation von zwei Probenserien mit Hilfe von Photolumineszenz-Spektroskopie untersucht. Die untersuchten Probenserien unterteilen sich in eine QW-Dicken-Variation und einer Serie mit Variation des Al-Gehalts in den QWs. 
Es zeigte sich, dass die Polarisation sich mit steigendem Al-Gehalt von TE- hin zu TM-polarisierten Licht ändert. Der Wechsel findet bei einer Wellenlänge von ca. $240 \thinspace nm$ statt und ist in guter Übereinstimmung mit den Simulation.
\newline
Die Ergebnisse der Polarisationsmessungen bei variierender QW-Dicke und festen Al-Gehalt bestätigen der Verlauf er Simulationen in Bezug auf den zu erwartenden Trend, wo am Rand zum Wechsel (bei $1$ und $4 \thinspace nm$) von TE zu TM der geringste Polarisationsgrad zu erwarten ist. 



