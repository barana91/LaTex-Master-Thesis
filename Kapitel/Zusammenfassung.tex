
\chapter*{Zusammenfassung}
\addcontentsline{toc}{chapter}{Zusammenfassung}
\thispagestyle{fancy}


In dieser Arbeit wurden PL-Untersuchungen an AlGaN UVC-Laser- und LED-Strukturen durchgeführt und deren IQE bestimmt. Diese wurde verglichen und die Ursachen für die Unterschiede mit Hilfe verschiedener Messmethoden wie AFM und CL versucht zu ergründen. In Kapitel \ref{chap:auger} wurde das Modell zur Bestimmung der IQE erweitert, in dem die Annahme es würde keine nicht-radiative Rekombination bei Tieftemperatur geben, fallen gelassen wurde. So findet Auger-Rekombination auch bei Tieftemperatur statt und hat durch die kubische Abhängigkeit von der Ladungsträgerdichte einen nicht zu vernachlässigenden Einfluss auf die IQE. Aus der Erweiterung folgen deutliche Korrekturen der IQE, die zu wesentlich geringeren Werten führen als das anfängliche Modell. Des Weiteren wurde der Einfluss verschiedener Effekte auf die Bestimmung der IQE untersucht. Die entscheidene Problematik dabei stellt die Überbandanregung dar, die es nicht erlaubt resonant anzuregen und zum Effekt der thermischen Diffusion führt. 
\newline
In Kapitel \ref{chap:mqw} wurde das MQW-Design von AlGaN UVC-LED Heterostrukturen untersucht, indem die IQEs von je vier Proben zweier Probenserien mit und ohne
dotierte Barrieren untersucht und verglichen wurden.  
Die Ergebnisse zeigen einen eindeutigen Einfluss der QW-Dicke auf die Rekombination und somit auf die IQE. Weiter zeigte sich der Einfluss dotierter Barrieren. So führen diese zu höheren IQEs im Bereich geringer Anregungsleistungsdichten. Die Messungen ergeben, dass QW-Dicken von $1-2 \thinspace nm$ in den höchsten IQEs resultieren. Die experimentell ermittelten Ergebnisse wurden von $k \cdot p$-Simulationen bestätigt. 
\newline
Die Untersuchungen in Kapitel \ref{chap:offcut} widmeten sich der Untersuchung zweier Probenreihen von optisch gepumpten Laserstrukturen, die aus Rezepten aus zwei unterschiedlichen Serien stammen. Untersucht wurde der Einfluss des Fehlschnittwinkels des Substrates. Ziel der Untersuchungen war, den Einfluss des Fehlschnittwinkels auf die IQE zu bestimmen. Für eine tiefergehende Analyse wurden AFM und CL-Messungen betrachtet, analysiert und mit der IQE in Zusammenhang gebracht. 
\newline
Es zeigte sich, dass die Proben mit Übergitter eine eindeutig glattere Oberfläche aufweisen. Ein positiver Einfluss auf die Versetzungsdichte bei einem Fehlschnittwinkel von $0.2 \degree$ konnte dagegen nicht eindeutig geklärt werden. Dazu wurde auf die bestehenden Grenzen bei der Bestimmung der IQE eingegangen und erläutert. Insgesamt konnte festgestellt werden, dass Übergitter mit einem Fehlschnitt von $0.1\degree$ aufgrund der glatteren Oberflächenmorphologie zu bevorzugen sind.
\newline
In Kapitel \ref{chap:pol} wurde die Polarisation und der aus Simulationen resultierende Kreuzungspunkt experimentell untersucht. Dazu wurde die Polarisation von zwei Probenserien mit Hilfe von Photolumineszenz-Spektroskopie bestimmt. Für die erste Serie wurde die QW-Dicke und für die zweite der Al-Gehalt in den QWs variiert.
Es zeigte sich, dass sich die Polarisation mit steigendem Al-Gehalt von TE- hin zu TM-polarisierten Licht ändert. Der Wechsel findet bei einer Wellenlänge von ca. $240 \thinspace nm$ statt und ist in guter Übereinstimmung mit den Simulationen.
\newline
Die Ergebnisse der Serie mit variierter QW-Dicke und festem Al-Gehalt bestätigen den Verlauf der Simulationen bezüglich des zu erwartenden Trends, wo am Rand zum Wechsel (bei $1$ und $4 \thinspace nm$) von TE zu TM der geringste Polarisationsgrad zu erwarten ist. 
\newline
Die in Kapitel \ref{chap:raum} überprüfte Methode zur Bestimmung der IQE bei Raumtemperatur liefert keine physikalisch verwertbaren Resultate. Das Modell scheitert daran, dass wichtige Einflüsse wie nicht-resonante Anregung, Dotierung, Screening und thermische Diffusion nicht miteinbezogen werden, aber speziell mit dem verwendeten Setup einen nicht zu vernachlässigenden Einfluss haben.  

