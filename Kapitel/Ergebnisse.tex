\chapter{Ergebnisse}
\thispagestyle{fancy}
\section{Untersuchung optisch gepumpter Laserstrukturen auf unterschiedlichen Templates}


Dieses Kapitel widmet sich der Untersuchung der beiden Probenreihen TS4045 und TS4048 von optisch gepumpten Laserstrukturen die aus Rezepten aus zwei unterschiedlichen Serien stammen. Die beiden Serien unterscheiden sich im wesentlichen dadurch, dass sie mit(TS4048) und ohne Übergitter(TS4045) gewachsen wurden. Jede Reihe für sich weist  zusätzlich noch Unterschiede den Proben selbst auf, so sind zwei Proben der Reihe TS4045 auf AlN-Bulk zweier unterschiedlicher Hersteller (HexaTech, IKZ) gewachsen und alle  anderen Proben auf ELO AlN/Sapphire mit jeweils 3 unterschiedlichen "offcut"-Winkeln. Tabellerisch sieht die  Zusammenstellung wie folgt aus: 

\vspace{1cm}


\setlength{\arrayrulewidth}{0.5mm}
\setlength{\tabcolsep}{0.5pt}
\renewcommand{\arraystretch}{1.5}
 

\begin{tabular}{ |c|c|c|c|c|c|   }
\hline
\multicolumn{3}{|c|}{TS4045} & \multicolumn{3}{c|}{TS4048}  \\
\hline
Endung & offcut& Template & Endung & offcut& Template \\
\hline
-2V* & 0.1$^\circ$m & ELO & -2V* & 0.1$^\circ$m & ELO \\
-2H & 0.1$^\circ$m & ELO & -2H & 0.1$^\circ$m & ELO \\
-2Z & 0.2$^\circ$m & ELO & -1 & 0.1$^\circ$m & ELO \\
-1 & 0.1$^\circ$m & Bulk(IKZ) & -2V* & 0.1$^\circ$m & ELO \\
-3* & 0.1$^\circ$m & Bulk(Hexatech) & -2V* & 0.1$^\circ$m & ELO \\
\hline



\end{tabular}


