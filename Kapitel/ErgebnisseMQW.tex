\thispagestyle{fancy}
\justifying

\section{Untersuchung des MQW-Designs von AlGaN-Heterostrukturen}
\label{sec:mqw}
%
\begin{figure}[H]
  \centering
  \begin{minipage}[t]{0.49\textwidth}
    \centering
    \includegraphics[width=\textwidth]{Bilder/MQWdickenSerie/undotiert}
		\caption{Aufbau der untersuchten MQW-Proben ohne dotierte Barrieren.}
    \label{fig:undotiert}
  \end{minipage}
	\hfill
  \begin{minipage}[t]{0.49\textwidth}
    \centering
    \includegraphics[width=\linewidth]{Bilder/MQWdickenSerie/dotiert}
		\caption{Aufbau der untersuchten MQW-Proben mit dotierten Barrieren.}
    \label{fig:dotiert}
  \end{minipage}
\end{figure}
\noindent 
% 
Dieses Kapitel widmet sich der Untersuchung von AlGaN-MQWs verschiedener Dicken mit dotierterten und undotierterten Barrieren. Mit Hilfe der UV-Photolumineszenz sollen die interne Quanteneffizienz und Wellenlänge hin bestimmt und der Einfluss des QCSE untersucht werden. 
Der Aufbau der Proben ohne dotierte Barrieren ist in Abbildung \ref{fig:undotiert} zu sehen.
Die aktive Zone setzt sich aus zwei $5$nm dicken $ Al_{0.81}Ga_{0.19}N$-Barrieren zwischen den drei $ Al_{0.60}Ga_{0.40}N$ QWs mit variierender Dicke $d$ mit $d=1,2,3,4 \thinspace nm$. Die aktive Zone befindet sich zwischen zwei $40 \thinspace nm$ dicken $ Al_{0.81}Ga_{0.19}N$-Barrieren von der eine die oberste Schicht darstellt. Darunter folgt eine AlN ($100\%$)-Bufferschicht die auf einem ELO-AlN/Saphir Substrat aufgewachsen wurde. 
Abbildung \ref{fig:dotiert} zeigt den Aufbau der Proben mit dotierten Barrieren. Die aktive Zone setzt sich aus zwei $5$nm dicken und Si-dotierten $ Al_{0.83}Ga_{0.17}N:Si$-Barrieren zwischen den drei $ Al_{0.72}Ga_{0.36}N$ QWs mit variierender Dicke $d$ mit $d=0.5,1.0,2.2,4 \thinspace nm$. Die aktive Zone befindet sich zwischen zwei $40 \thinspace nm$ dicken und Si-dotierten $ Al_{0.81}Ga_{0.19}N:Si$-Barrieren von der eine die oberste Schicht darstellt. Darunter folgt eine AlN ($100\%$)-Bufferschicht die auf einem ELO-AlN/Saphir Substrat aufgewachsen wurde. 
%
\begin{figure}[H]
  \centering
  \begin{minipage}[t]{0.49\textwidth}
    \centering
    \includegraphics[width=\textwidth]{Bilder/MQWdickenSerie/spektrumUndotiert}
		\caption{PL-Spektrum der untersuchten MQW-Proben ohne dotierte Barrieren.}
    \label{fig:undotiertSpektrum}
  \end{minipage}
	\hfill
  \begin{minipage}[t]{0.49\textwidth}
    \centering
    \includegraphics[width=\linewidth]{Bilder/MQWdickenSerie/spektrumDotiert}
		\caption{PL-Spektrum der untersuchten MQW-Proben mit dotierten Barrieren.}
    \label{fig:dotiertSpektrum}
  \end{minipage}
\end{figure}
\noindent 
% 
In den PL-Spektren in den Abbildungen \ref{fig:undotiertSpektrum} und \ref{fig:dotiertSpektrum}
ist zu erkennen, dass mit sinkender QW-Dicke $d$ die Emissionsenergie steigt. Die Gründe hierfür sind der QCSE und der Ladungsträgereinschluss (engl.: Confinement). Die eingeschlossenen Ladunsgträger führen zum Screening und dieser Effekt wiederum ist stärker bei kleineren Dicken des QWs. Zudem ist ersichtlich, dass die Barrierenemission von der Emission der QWs mit steigender QW-Dicke stärker voneinander getrennt werden. Grund hierfür ist, dass der QCSE und der Ladungsträgereinschluss nur einen Einfluss auf die Emissionseigenschaften der QWs haben???.
%
\begin{figure}[H]
  \centering
  \begin{minipage}[t]{0.49\textwidth}
    \centering
    \includegraphics[width=\textwidth]{Bilder/MQWdickenSerie/intTTundotiert.png}
		\caption{Integrierte Intensität in doppeltlogarithmischer Darstellung in Abhängigkeit der Anregungsleistungsdichte bei Tieftemperatur für die Proben mit undotierter Barriere.}
    \label{fig:undotiertint}
  \end{minipage}
	\hfill
  \begin{minipage}[t]{0.49\textwidth}
    \centering
    \includegraphics[width=\linewidth]{Bilder/MQWdickenSerie/intTTdotierte.png}
		\caption{Integrierte Intensität in doppeltlogarithmischer Darstellung in Abhängigkeit der Anregungsleistungsdichte bei Tieftemperatur für die Proben mit dotierter Barriere.}
    \label{fig:dotiertint}
  \end{minipage}
\end{figure}
\noindent 
% 
In den Abbildungen \ref{fig:undotiertint} und \ref{fig:dotiertint} ist die integrierte Intensität in Abhängigkeit der Anregungsleistungsdichte bei Tieftemperatur für die Proben mit undotierter und dotierter Barriere in doppeltlogarithmischer Darstellung dargestellt.
Im Bereich geringer Anregungsleistungsdichten zeigt sich für beide Serien eine lineare Steigung (gestrichelte Linie) die mit zunehmender Anregungsleistungsdichte ein nichtlinearen Verlauf annimmt der auf Auger-Rekombination zurückzuführen ist (siehe Kapitel \ref{chap:auger}).
Die integrierten Intensitäten befinden sich für die Proben mit undotierter Barriere bei Tieftemperatur für alle QW-Dicken auf dem selben Niveau.
Die Proben mit einer QW-Dicke von $3,0 \thinspace nm$ und $4,0 \thinspace nm$ leuchten am hellsten und darauf folgen die Proben mit $2.0 \thinspace nm$ und $1.0 \thinspace nm$  QW-Dicke. Die Unterschiede sind aber marginal und im Rahmen der Streuung, die durch die unterschiedlichen Fokussierungen und Position im PL-Aufbau zu erwarten sind. Gleiches gilt für die Proben mit dotierter Barriere. Dort fallen die Proben mit QW-Dicken von $2.2 \thinspace nm$ und $4.4 \thinspace nm$ deutlicher ab, jedoch in einem unbedenklichen Maß.   
%
\begin{figure}[H]
  \centering
  \begin{minipage}[t]{0.49\textwidth}
    \centering
    \includegraphics[width=\textwidth]{Bilder/MQWdickenSerie/PeakEnergieUndotiert.png}
		\caption{Peak Emissionsenergie in Abhängigkeit der Anregungsleistungsdichte bei Tieftemperatur der untersuchten MQW-Proben ohne dotierte Barrieren.}
    \label{fig:undotiertpeak}
  \end{minipage}
	\hfill
  \begin{minipage}[t]{0.49\textwidth}
    \centering
    \includegraphics[width=\linewidth]{Bilder/MQWdickenSerie/PeakEnergieDotiert.png}
		\caption{Peak Emissionsenergie in Abhängigkeit der Anregungsleistungsdichte bei Tieftemperatur der untersuchten MQW-Proben mit dotierten Barrieren.}
    \label{fig:dotiertpeak}
  \end{minipage}
\end{figure}
\noindent 
% 
Die Abbildungen \ref{fig:undotiertpeak} und \ref{fig:dotiertpeak} zeigen die Peak Emissionsenergie in Abhängigkeit der Anregungsleistungsdichte bei Tieftemperatur für die Proben ohne und mit dotierten Barrieren bei Tieftemperatur. Für die Proben mit undotierten Barrieren ist zu erkennen, dass mit steigender Anregungsleistungsdichte die Emissionsenergie steigt. Der Grund hierfür ist das Screening des QCSE. Dieser Effekt wird größer mit kleiner werdender QW-Dicke, weil mit sinkender QW-Dicke immer weniger Ladungsträger nötig sind um Screening zu erreichen. So ist die Verschiebung der Peak-Emissionswellenlänge bei der Probe mit einer QW-Dicke von $5 \thinspace nm$ von $4,65 \thinspace eV$ bei der geringsten Anregungsleistungsdichte zu $4,77 \thinspace eV$ bei der höchsten Anregungsleistungsdichte am stärksten. Die Probe mit einer QW-Dicke von $1 \thinspace nm$ zeigt dagegen, keinen Peak-Wellenlängen-Verschiebung. 
Die Proben mit dotierten Barrieren zeigt dagegen ein anderes Verhalten, so sind die Peak-Emissionsenergien für die Proben mit einer QW-Dicke von 
$1,0 \thinspace nm$, $2,2 \thinspace nm$ und $4,4 \thinspace nm$ nahezu konstant über den ganzen Bereich der Anregungsleistungsdichte, aber die Peak-Emissionsenergie der Probe mit der geringsten QW-Dicke von $0,5 \thinspace nm$ verschiebt sich mit steigender Anregungsleistungsdichte hin zu kleineren Energie. Die Gründe für dieses Verhalten sind bisher unklar.
%
\begin{figure}[H]
  \centering
  \begin{minipage}[t]{0.49\textwidth}
    \centering
    \includegraphics[width=\textwidth]{Bilder/MQWdickenSerie/IQEundotiert.png}
		\caption{IQE in Abhängigkeit der Anregungsleistungsdichte bei Raumtemperatur der untersuchten MQW-Proben ohne dotierte Barrieren.}
    \label{fig:undotiertIQE}
  \end{minipage}
	\hfill
  \begin{minipage}[t]{0.49\textwidth}
    \centering
    \includegraphics[width=\linewidth]{Bilder/MQWdickenSerie/IQEdotiert.png}
		\caption{IQE in Abhängigkeit der Anregungsleistungsdichte bei Raumtemperatur der untersuchten MQW-Proben.}
    \label{fig:dotiertIQE}
  \end{minipage}
\end{figure}
\noindent 
% 
Die Ergebnisse der IQE bei Raumtemperatur für die Proben ohne und mit dotierten Barrieren sind in den Abbildungen 
\ref{fig:undotiertIQE} und \ref{fig:dotiertIQE} zu sehen. Für beide Probenserien zeigt sich, dass aufgrund des QCSE die IQE für weite QWs 
($3.0 -4.0  nm$) geringer ausfällt und für besonders geringe Dicken ($0.5\thinspace nm$), wahrscheinlich durch den schlechteren Ladungsträgereinschluss, ebenfalls. Die höchste IQE wird für QW-Dicken von $1.0 \thinspace nm- 2.2 \thinspace nm$ gemessen. Die Proben ohne dotierte Barrieren haben eine IQE ca. $0,145$ bei höchster Anregungsleistungsdichte für die Proben mit QW-Dicken von $1.0 \thinspace nm$ und $2.0\thinspace nm$. 
Die Proben mit dotierten Barrieren haben eine IQE von ca. $0,121$ bei höchster Anregungsleistungsdichte für die Proben mit QW-Dicken von $1.0 \thinspace nm$ und $2.2\thinspace nm$. Anhand der Ergebnisse für die Proben mit dotierten Barrieren ist zu erkennen, dass die Si-dotierung einen deutlichen Einfluss auf den Verlauf der IQE hat, so ist die Ordinate der IQEs für diese Proben deutlich höher, wie Abbildung \ref{fig:dotiertIQE} zeigt. Die Dotierung führt sichtbar zu höheren IQEs bei geringen Anregungsleistungsdichten die Steigung mit steigender Anreungsleistungsdichte ist aber im Vergleich geringer, sodass die IQEs für beide Serien ungefähr den selben Wert einnehmen.
%
\begin{figure}[H]
  \centering
  \begin{minipage}[t]{0.49\textwidth}
    \centering
    \includegraphics[width=\textwidth]{Bilder/MQWdickenSerie/Simu1.png}
		\caption{Simulation der Elektron- und Lochwellenfunktion im Bändermodell mit QW in einem Bereich von $-4$ bis $4 \thinspace nm$ für verschiedene Dicken. Simulation von Christoph Reich.}
    \label{fig:undotiertSpektrum}
  \end{minipage}
	\hfill
  \begin{minipage}[t]{0.49\textwidth}
    \centering
    \includegraphics[width=\linewidth]{Bilder/MQWdickenSerie/Simu2.png}
		\caption{Wellenfunktion Überlappintegral in Abhängigkeit der QW-Dicke. Simulation von Christoph Reich.}
    \label{fig:dotiertSpektrum}
  \end{minipage}
\end{figure}
\noindent 
% 
Abbildung und zeigen Modellberechnungen von Christoph Reich basierend auf der k$\cdot$p-Theorie für variierende QW-Dicken. Die Simulation bestätigt, dass für eine QW-Dicke von $0,5 \thinspace nm$ die IQE wegen geringem Ladungsträgereinschluss gering ausfällt und dass durch den QCSE die Separation von Elektron- und Lochwellenfunktion steigt. Dies führt zur Reduktion der radiativen Rekombinationsrate und bedeutet, dass die IQE für weite QWs geringer ausfällt. Die Simulation bestätigt ebenfalls, dass QW-Dicken zwischen $1.0 \thinspace nm- 2.2 \thinspace nm$ die beste Grundlage für die Verwendung in LEDs darstellen.

\subsection{Zusammenfassung}

Die Ergebnisse dieses Kapitels zeigen, dass die QW-Dicke einen eindeutigen Einfluss auf die Rekombination und damit auf die IQE hat. Weiter wurde gezeigt, dass die Emissionsenergie mit abnehmender QW-Dicke steigt. Die Gründe dafür sind der Ladungsträgereinschluss und der QCSE und wurde von dem Simulationen bestätigt. Für die Proben mit undotierten Barrieren wurde gezeigt, dass die Emissionsenergie steigt mit steigender Anregungsleistungsdichte durch das Screening des QCSE durch zunehmende Anzahl von Ladungsträgern. Die Proben mit dotierten Barrieren zeigen dagegen ein anderes Verhalten, dessen Ursprung nicht geklärt werden konnte.   