
\chapter{Einleitung}
\thispagestyle{fancy}

\begin{quote}
In the spirit of Alfred Nobel the Prize rewards an invention of greatest benefit to mankind; using blue LEDs, white Light can be created in a new way.\end{quote}
Dieser Satz, den die Schwedische Akademie der Künste nach der Vergabe des Nobelpreises an die Entwicklung der blauen LED(kurz, light emitting diode) im Jahr 2014 an die Presse veröffentlichte, fasst treffend zusammen, wie hoch die Bedeutung der auf Halbleiterkristallen basierenden optischen Bauelemente ist.
LEDs nehmen einen fundamentalen und immer bedeutender werdenden Teil unseres alltäglichen Lebens ein. Ausgezeichnet durch ihre hervorragende Effizienz, konkurrenzlosen Lebensdauer und geringen Dimension übernimmt sie durch eine immer höher werdenden Lichtausbeute zusehends neue Anwendungsbereiche. 
%Seit jeher etabliert in den Bereichen der optischen Datenübertragung und Leuchtanzeige schreiten immer mehr andere Wellenlängenbereiche in den Fokus der weltweiten Forschung. 
Insbesondere auf Gallium Nitrid (GaN) basierende Halbleitermaterialien haben einen bahnbrechenden Weg hingelegt, der zur Entwicklung von hoch effizienten und leuchtstarken blauen LEDs führte und ebenfalls Grundlage für die Entwicklung in andere hochenergetische Wellenlängenbereiche darstellt~\cite{risk}.
So ebnet GaN auch den Weg für die Erzeugung von ultraviolet emittierenden Leuchtdioden. Der ultraviolette Spektralbereich, der sich unterteilt in den UV-A (400 nm bis 320 nm), UV-B (320 nm bis 280) und UV-C Bereich (280 nm bis 200 nm) ist bedeutend für eine sehr hohe Anzahl spezieller Anwendungsbereiche. Beispielsweise bieten sich UV-Leds an die bisher für Wasseraufbereitung genutzten Quecksilberdampflampen zu ersetzen, für deren Betrieb Hochspannungsnetzteile verwendet werden, die einen mobilen Einsatz erheblich erschweren können. Hier könnten UV-LEDs Abhilfe verschaffen, die durch ihr kleines Format und durch die niedrigen Betriebsspannungen einen Mobileneinsatz ermöglichen. Ein weiteres Anwendungsgebiet ist die industrielle Aushärtung/Aufbrechung von Lacken und die Gasdetektion. 
\newline
Für all diese möglichen Applikation ist eine hohe Ausgangsleistung notwendig. Aber wie im Graph [] zu sehen ist, sinkt die spektrale Emissonsleistung mit kleiner werdendem Wellenlängenbereich signifikant. Der Grund dafür ist, dass UV-LEDs an einer geringen Effizienz leiden. Quantitativ ausgedrückt Externe Quanteneffizienz (EQW. Die Gründe hierfür sind vielfältig. LEDs bestehen aus einer Viezahl an Schichten, die unterschiedlichen Funktionen dienen. Diese Schichten werden auf Substraten aufgewachsen. Daher ist eine hohe Substratqualität für die optischen Eigenschaften entscheidend. Eine geringe Defektdichte im Substrat geht einher mit einer ebenfalls geringen Defektdichte in den aufgewachsenen Schichten. Ein weiteres Problem im Zusammenhang mit den geringen Defektdichten, ist ein Mangel an geeigneten Substratmaterialien. So wird aufgrund des Mangels an AlN Substrate, beruhend auf den Schwierigkeiten bei der Herstellung, auf Saphir Substrate ausgewichen. Diese sind im fernen UV transparent und zusätzlich in großen Mengen in guter Qualität hergestellt.
Problematisch ist allerdings, die hohe Gitterfehlanpassung durch die relativ großen Unterschiede zwischen den Gitterkonstanten von AlN/GaN und Saphir.
Durch die hohe Gitterfehlanpassung, sind AlN- und AlGaN-Schichten nicht vollverspannt aufwachsbar. Das wiederum bedeutet die Schichten relaxieren, wenn die Elastitzät der Schicht nicht groß genug im Vergleich zur Verspannungsenergie ist. Die Relaxation führt zur Entstehung von Versetzungen und Rissen. Diese agieren im Kristall als sogenante nicht-radiative Rekombinationszentren
die die interne Quanteneffizienz (IQE) verringern. Mit Hilfe von temperaturabhängigen Photolumineszenzmessungen kann diese bestimmt werden. 









