
\chapter{Einleitung}
\thispagestyle{fancy}

\begin{quote}
In the spirit of Alfred Nobel the Prize rewards an invention of greatest benefit to mankind; using blue LEDs, white Light can be created in a new way.\end{quote}
Dieser Satz den die Schwedische Akademie der Künste nach der Vergabe des Nobelpreises an die Entwicklung der blauen LED(kurz, light emitting diode) im Jahr 2014 an die Presse veröffentlichte, fasst treffend zusammen, wie hoch die Bedeutung der auf Halbleiterkristallen basierenden optischen Bauelemente ist.
LEDs nehmen einen fundamentalen und immer bedeutender werdenden Teil unseres alltäglichen Lebens ein. Ausgezeichnet durch ihre hervorragende Effizienz, konkurrenzlosen Lebensdauer und geringen Dimension übernimmt sie durch eine immer höher werdenden Lichtausbeute zusehends neue Anwendungsbereiche. 
%Seit jeher etabliert in den Bereichen der optischen Datenübertragung und Leuchtanzeige schreiten immer mehr andere Wellenlängenbereiche in den Fokus der weltweiten Forschung. 
Insbesondere auf Gallium Nitrid (GaN) basierende Halbleitermaterialien haben einen bahnbrechenden Weg hingelegt, der zur Entwicklung von hoch effizienten und leuchtstarken blauen LEDs führte.


