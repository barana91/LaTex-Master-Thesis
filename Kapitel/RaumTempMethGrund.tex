
\section{Bestimmung der IQE bei Raumtemperatur durch Fitting}

\thispagestyle{fancy}

In diesem Kapitel soll nun eine in \cite{doi:10.1063/1.3100773} und \cite{doi:10.1063/1.4917540} gezeigte Methode zur Bestimmung der IQE durch ein Fitting-Modell für die integrierte Intensität in Abhängigkeit der Ladungsträgerdichte vorgestellt werden. Angefangen mit der Rekombinationasrate, geht das Modell davon aus,
das bei Raumtemperatur Auger-Rekombination nur bei sehr hohen Anregungsleistungsdichten Relevant ist, wegen der kubischen Abhängigkeit der Auger-Rekombination von der Ladungsträgerdichte $n$
\begin{equation}
    G = R_{eff} = A \cdot n + B \cdot n^2
\end{equation}    
G steht hierbei für Generationsrate und beschreibt namentlich die Rate der Ladungsträger die durch Bestrahlung mit dem Laser erzeugt werden und entspricht hierbei der effektiven Rekombinationsrate $R_{eff}$


