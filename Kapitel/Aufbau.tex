
\chapter{Aufbau}


\thispagestyle{fancy}

\section{Photolumineszenzaufbau}

Für die experimentelle Untersuchung der UV-Photolumineszenzaufbau wurde der PL-Aufbau der AG-Kneissl verwendet, den Christoph Reich in der Zeit seiner Masterarbeit aufgebaut und während seiner Promotion erweitert hat\cite{creich}. 
Als Anregungsquelle für die Photolumineszenz dient ein ArF-Excimerlaser mit einer Wellenlänge von $193 \ nm$ ($6,4 \ eV$). Mit dieser Wellenlänge ist er bestens geeignet für die Überbandanregung von Nitridhalbleitern. 
Des Weiteren bietet der Aufbau die Möglichkeit von temperaturabhängige Untersuchungen von $5 \ K $ bis 300 K. Dies ist auch die Grundlage Bestimmung der Internen Quanteneffizienz (kurz IQE) dar, die den Großteil der Thematik dieser Arbeit ausmachen wird. Der Laser mit dem Modellnamen  "Xantos" von der Firma Coherent bietet eine maximale Ausgangsleistung von $ 5000 \ \frac{kW}{cm^2} $ und die Frequenz ist bis zu 500 Hz einstellbar bei einer Pulsdauer von $5 \ ns$. Durch interne Rückkopplung ist eine Energiestabilisierung möglich, die die Schwankung der Anregungsleistung auf 3 Prozent minimiert. 
\newline




